%%%%%%%%%%%%%%%%%%%%%%%%%%%%%%%%%%%%%%%%%
% Medium Length Professional CV
% LaTeX Template
% Version 2.0 (8/5/13)
%
% This template has been downloaded from:
% http://www.LaTeXTemplates.com
%
% Original author:
% Trey Hunner (http://www.treyhunner.com/)
%
% Important note:
% This template requires the resume.cls file to be in the same directory as the
% .tex file. The resume.cls file provides the resume style used for structuring the
% document.
%
%%%%%%%%%%%%%%%%%%%%%%%%%%%%%%%%%%%%%%%%%

%----------------------------------------------------------------------------------------
%	PACKAGES AND OTHER DOCUMENT CONFIGURATIONS
%----------------------------------------------------------------------------------------

\documentclass{resume} % Use the custom resume.cls style

\usepackage[left=0.75in,top=0.6in,right=0.75in,bottom=0.6in]{geometry} % Document margins

\name{Cirrus (Fangyu) Gai} % Your name
\address{greferry@gmail.com} % Your phone number and email

\begin{document}

%----------------------------------------------------------------------------------------
%	EDUCATION SECTION
%----------------------------------------------------------------------------------------

\begin{rSection}{Employment}

{\bf Babylon Labs} \hfill {\em May 2022 - Present} \\ 
{\em Staff Research Engineer, full-time} \hfill {\em Remote}

{\bf Alibaba DAMO Academy} \hfill {\em Apr. 2021 - Apr. 2022} \\ 
{\em Research \& Development Intern, full-time} \hfill {\em Hangzhou, China}

{\bf Hangzhou Qulian Technology} \hfill {\em Sep. 2020 - Apr. 2021} \\ 
{\em Research \& Development Intern, full-time} \hfill {\em Hangzhou, China}

{\bf Dapper Labs} \hfill {\em June 2019 - Mar. 2020} \\ 
{\em Research \& Development Intern, part-time} \hfill {\em Vancouver, Canada}

\end{rSection}

\begin{rSection}{Education}

{\bf University of British Columbia} \hfill {\em Sep. 2018 - Sep. 2022} \\ 
{\em Ph.D. in Electrical Engineering} \hfill {\em Kelowna \& Vancouver, Canada}\\
Doctoral Student member of Blockchain@UBC \\
Advisor: Chen Feng, also closely working with Ivan Beschastnikh\\
Dissertation: On the Performance of Byzantine Fault-tolerant Consensus in the Blockchain Era \smallskip

{\bf Beijing Institute of Technology} \hfill {\em Sep. 2011 - June 2015} \\ 
{\em B.Eng. in Software Engineering} \hfill {\em Beijing, China}\\
Advisor: Zhiqiang Li, Hongchen Guo\\
Dissertation: Enhance Adaboost Algorithm by Integrating LDA Topic Model\\
\em Ranking: 2nd out of 39 in the direction of Information Security

\end{rSection}

\begin{rSection}{Research Interests}
Blockchain, Databases, Distributed Systems, Internet of Things
\end{rSection}

%----------------------------------------------------------------------------------------
%	WORK EXPERIENCE SECTION
%----------------------------------------------------------------------------------------

\begin{rSection}{Experience}

\begin{rSubsection}{Babylon Labs}{\em May 2022 - Present}{Staff Research Engineer}{remote}
\item Research and development of the Babylon node and auxiliary programs including finality provider, vigilante, and indexer.
\item Technical lead of Babylon BitVM bridge.
\\Website: https://babylonlabs.io.
\\Project link: https://github.com/babylonlabs-io.
\end{rSubsection}

\begin{rSubsection}{Alibaba DAMO Academy}{\em Apr. 2021 - Apr. 2022}{Research \& Development Intern. Mentor: Sheng Wang.}{Hangzhou, China}
\item Research on scaling BFT consensus protocols by designing a robust shared mempool.
\\Project link: https://github.com/gitferry/bamboo-stratus.
\item Research on enhancing the centralized ledger database (LedgerDB) with ubiquitous verification abilities.
\\Project link: https://www.alibabacloud.com/product/ledgerdb.
% \item Research on applying TEE trusted hardware (e.g., SGX) in building a confidential blockchain.
\end{rSubsection}

%------------------------------------------------

\begin{rSubsection}{Hangzhou Qulian Technology}{\em Sep. 2020 - Apr. 2021}{Research \& Development Intern. Mentor: Hao Duan.}{Hangzhou, China}
\item Research on the performance of state-of-the-art BFT protocols for permissioned blockchains.
\\Project link: https://github.com/gitferry/bamboo.
\item Re-designed and implemented some of the core components (e.g., Raft and Mempool) of the consensus services in Hyperchain.
% \item Leverage trusted hardware (e.g., SGX) to improve the scalability of BFT consensus.
\end{rSubsection}

%------------------------------------------------

\begin{rSubsection}{Dapper Labs}{\em June 2019 - Mar. 2020}{Research \& Development Intern. Mentor: Alexander Hentschel.}{Vancouver, Canada}
\item Explored the wide design space of blockchain consensus protocols, e.g., HotStuff, Algorand.
\item Built demos of the HotStuff BFT protocol and evaluated its performance.
\item Participated in building the consensus node of the Flow blockchain.
\\Project link: https://github.com/onflow/flow-go.
\end{rSubsection}

\begin{rSubsection}{University of British Columbia}{\em Winter 2018 \& Winter 2020}{Teaching \& Research Assistant}{Vancouver, Canada}
\item ENGR 453: Internet of Things.
\item ENGR 464: Distributed Ledger Technologies with Engineering Applications.
\end{rSubsection}

\end{rSection}

%----------------------------------------------------------------------------------------
%	TECHNICAL STRENGTHS SECTION
%----------------------------------------------------------------------------------------

\begin{rSection}{Technical Strengths}

\begin{tabular}{ @{} >{\bfseries}l @{\hspace{6ex}} l }
Computer Languages & Golang, Rust, Python \\
Protocols \& Blockchain Platform & Bitcoin, Cosmos, BitVM, CometBFT, HotStuff, etcd \\
Tools & Vim, git, Markdown, \LaTeX\\
\end{tabular}

\end{rSection}

\begin{rSection}{Languages}
Mandarin (Native), English (Professional), Japanese (Basic)
\end{rSection}

\begin{rSection}{Selected Publication}

\begin{rSubsection}{Conference Papers}{}{}{}
\item [C10] Ertem Nusret Tas, David Tse, \textbf{Fangyu Gai}, Sreeram Kannan, Mohammad Ali Maddah-Ali, Fisher Yu, ``Bitcoin-Enhanced Proof-of-Stake Security: Possibilities and Impossibilities'' in \textit{Proc. IEEE S\&P, 2023}.
\textbf{CCF-A}.

\item [C9] \textbf{Fangyu Gai}, Jianyu Niu, Ivan Beschastnikh, Chen Feng, Sheng Wang, ``Scaling Blockchain Consensus via a Robust Shared Mempool'' in \textit{Proc. IEEE ICDE, 2023}.
\textbf{CCF-A}.

\item [C8] Xinying Yang, Sheng Wang, Feifei Li, Yuan Zhang, Wenyuan Yan, \textbf{Fangyu Gai}, Benquan Yu, Likai Feng, Qun Gao, and Yize Li, ``Ubiquitous Verification in Centralized Ledger Database'' in \textit{Proc. IEEE ICDE, 2022}.
\textbf{CCF-A}.

\item [C7] Hanzheng Lyu, Jianyu Niu, \textbf{Fangyu Gai}, and Chen Feng, ``Publish or Perish: Defending Withholding Attack in Dfinity Consensus'' in \textit{Proc. IEEE MSN, 2021} (\textbf{best paper award}).
\textbf{CCF-C}.

\item [C6] \textbf{Fangyu Gai}, Jianyu Niu, Seyed Ali Tabatabaee, Chen Feng, and Mohammad Mussadiq Jalalzai, ``Cumulus: A Secure BFT-based Sidechain for Off-chain Scaling'' in \textit{Proc. IEEE IWQoS, 2021}.
\textbf{CCF-B}.

\item [C5] \textbf{Fangyu Gai}, Ali Farahbakhsh, Jianyu Niu, Chen Feng, Ivan Beschastnikh, and Hao Duan, ``Dissecting the Performance of Chained-BFT'' in \textit{Proc. IEEE ICDCS, 2021}.
\textbf{CCF-B}.

\item [C4] Jiangyu Niu, \textbf{Fangyu Gai}, Mohammad Mussadiq Jalalzai, and Chen Feng, ``On the Performance of Pipeline HotStuff'' in \textit{Proc. IEEE INFOCOM, 2021}.
\textbf{CCF-A}.

\item [C3] Jianyu Niu, Ziyu Wang, \textbf{Fangyu Gai}, and Chen Feng, ``Incentive Analysis of Bitcoin-NG, Revisited'' in \textit{Proc. IFIP $WG_{7.3}$ Performance, 2020.}
\textbf{CCF-B}.

\item [C2] \textbf{Fangyu Gai}, Baosheng Wang, Wenping Deng, and Wei Peng, ``Proof of Reputation: A Reputation-Based Consensus Protocol for Peer-to-Peer Network'' in \textit{Proc. DASFAA, 2018.}
\textbf{CCF-B}.

\item [C1] Dongxing Li, Wei Peng, Wenping Deng, \textbf{Fangyu Gai}, ``A Blockchain-Based Authentication and Security Mechanism for IoT'' in \textit{Proc. ICCCN, 2018.}
\textbf{CCF-C}.

\end{rSubsection}

\begin{rSubsection}{Journal Papers}{}{}{}
\item [J4] Mohammad Jalalzai, Jianyu Niu, Chen Feng, and \textbf{Fangyu Gai}, ``Fast-Hotstuff: A Fast and Resilient Hotstuff Protocol'' in \textit{IEEE TDSC, 2024.}
\textbf{CCF-A}.

\item [J3] \textbf{Fangyu Gai}, Jianyu Niu, Seyed Ali Tabatabaee, Chen Feng, and Mohammad Mussadiq Jalalzai, ``A Secure Sidechain for Decentralized Trading in Internet of Things'' in \textit{IEEE IoT Journal, 2024}.
\textbf{JCR-Q1}.

\item [J2] Jianyu Niu, \textbf{Fangyu Gai}, and Chen Feng, ``Crystal: Enhance Blockchain Mining Transparency with Quorum Certificat'' in \textit{IEEE TDSC, 2023.}
\textbf{CCF-A}.

\item [J1] Jianyu Niu, Ziyu Wang, \textbf{Fangyu Gai}, and Chen Feng, ``Incentive Analysis of Bitcoin-NG, Revisited'' in \textit{PEVA, 2020.}
\textbf{CCF-B}.

\end{rSubsection}

% \begin{rSubsection}{In Submission}{}{}{}
% \item [I3] \textbf{Fangyu Gai}, Jianyu Niu, Seyed Ali Tabatabaee, Chen Feng, and Mohammad Mussadiq Jalalzai, ``A Secure Sidechain for Decentralized Trading in Internet of Things'' submitted to \textit{IEEE IoT Journal}.
% \item [I2] Jianyu Niu, \textbf{Fangyu Gai}, and Chen Feng, ``Crystal: Enhance Blockchain Mining Transparency with Quorum Certificat'' submitted to \textit{IEEE TDSC.} \textbf{CCF-A}.
% \item [I1] Mohammad Jalalzai, Jianyu Niu, Chen Feng, and \textbf{Fangyu Gai}, ``Fast-Hotstuff: A Fast and Resilient Hotstuff Protocol'' submitted to \textit{IEEE TDSC.} \textbf{CCF-A}.
% \end{rSubsection}


\end{rSection}

%----------------------------------------------------------------------------------------
%	EXAMPLE SECTION
%----------------------------------------------------------------------------------------

%\begin{rSection}{Section Name}

%Section content\ldots

%\end{rSection}

%----------------------------------------------------------------------------------------

\end{document}
